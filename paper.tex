\documentclass[a4paper,10pt]{article}
\usepackage[utf8]{inputenc}

\usepackage{hyperref}
\usepackage{multirow}
\usepackage{cite}
\usepackage{algorithm}
\usepackage[noend]{algorithmic}
\usepackage{xspace}
\usepackage{enumerate}
\usepackage{array} 
\usepackage{color}
% \usepackage{subcaption}
\usepackage{wrapfig}
\usepackage{subfig}
\usepackage{suffix}
% \usepackage{hhline}
% \usepackage{cases}
\usepackage{amsfonts}
\usepackage{multirow}
% \usepackage{ltlfonts}
% \usepackage{myutils}
\usepackage{thmtools}
\usepackage{mathtools}
% \usepackage{thmtools}
\usepackage{placeins}
\usepackage{changepage}
% \usepackage{subfig}
% \usepackage{tikz}
% \usetikzlibrary{arrows,shapes, matrix,calc,positioning}
% \usepackage{times}            
% \usepackage{latexsym}
\usepackage{listings}
\usepackage{verbatim}
\usepackage{alltt}
\usepackage{float}
% \usepackage{calc}
% \let\proof\relax
% \let\endproof\relax
% \usepackage{amsthm}

\usepackage{amsmath}
% \usepackage{setspace}
% \usepackage{relsize}
% \usepackage{pdfpages}
% \usepackage{myutils}
\usepackage{graphicx}
% \usepackage{cite}
\usepackage{latexsym}
\usepackage{amssymb}
\usepackage{proof}
\usepackage{amsfonts}
% \usepackage{mathpartir}


\usepackage{tikz}
\usetikzlibrary{arrows,shapes, matrix,calc,positioning,chains}

% % % % % % % % % % % % % % % % % % % % % % % % % % % % % % % % % % % % % % % % % % % % % % % 
% % % % % % % % % % % % % % % % % % % % % % % % % % % % % % % % % % % % % % % % % % % % % % % 
% % % % % % % % % % % % % % % % % % % % % % % % % % % % % % % % % % % % % % % % % % % % % % % 
\newcommand{\etal}{\textit{et al.}\xspace}
\newcommand{\ie}{\textit{i.e.}\xspace}
\newcommand{\eg}{\textit{e.g.}\xspace}
\newcommand{\viz}{\textit{viz.}\xspace}
\newcommand{\etc}{\textit{etc.}\xspace}
\newcommand{\state}{\ensuremath{\pi}\xspace}
\newcommand{\statedown}[1]{\ensuremath{\pi_{#1}}\xspace}
\newcommand{\stateset}{\ensuremath{\Pi}\xspace}
\newcommand{\policyfunc}{\ensuremath{\Psi}\xspace}
\newcommand{\planguage}{\ensuremath{\mathbf{\mathcal{AAPL}}}\xspace}
\newcommand{\Paragraph}[1]{\vspace{2pt}\noindent\textbf{\textit{#1}}}
\newcommand{\cD}{\ensuremath{\mathcal{D}}\xspace}
\newcommand{\cR}{\ensuremath{\mathcal{R}}\xspace}
\newcommand{\cA}{\ensuremath{\mathcal{A}}\xspace}
\newcommand{\funcname}[1]{\textbf{\texttt{#1}}}
\newcommand{\dom}{\ensuremath{\Delta}\xspace} %domain
\newcommand{\domop}{\funcname{opDom}\xspace} %domain
\newcommand{\OP}{\ensuremath{\mathcal{OP}}\xspace}
\newcommand{\arity}{\ensuremath{\iota}\xspace}
\newcommand{\entity}{\ensuremath{\lambda}\xspace}
\newcommand{\Allow}{\ensuremath{\mbox{Allow}}\xspace}
\newcommand{\Deny}{\ensuremath{\mbox{Deny}}\xspace}
\newcommand{\op}{\ensuremath{\mu}\xspace}
\newcommand{\abac}{\ensuremath{\Upsilon}\xspace}

% % % % % % % % % % % % % % % % % % % % % % % % % % % % % % % % % % % % % % % % % % % % % % % 
% % % % % % % % % % % % % % % % % % % % % % % % % % % % % % % % % % % % % % % % % % % % % % % 
% % % % % % % % % % % % % % % % % % % % % % % % % % % % % % % % % % % % % % % % % % % % % % % 


%opening
\title{Towards A Family of Attribute-based Access Control Models and its Policy Specification Languages}
\author{}
\date{}
\begin{document}

\maketitle

\begin{abstract}

\end{abstract}

\section{An Abstract Attribute-Based Access Control Model (\abac)}
Formally our abstract model, which we denote by \abac, has the following form 
$\abac=\langle\cA, \cD, \dom, \OP, \state, \policyfunc\rangle$. 
We have an unbounded number of \emph{attribute names} (\eg, user-name, role), the set of which is denoted by, $\cA=\{a_1, a_2,\ldots\}$.  
For each attribute name $a_i\in\cA$ where $i\in\mathbb{N}$, we have an associated, possibly infinite, domain $d_{a_i}$ (\eg, $\mathbb{N}$, 
string, set of strings, list) from which $a_i$ can assume values from. 
The set of all such domains are represented by $\cD=\{d_{a_1}, d_{a_2}, \ldots\}$.  Given an attribute 
name $a_i \in \cA$, we have a total map, denoted by \dom, that maps attribute names to their 
appropriate domain. Formally, $\dom: \cA \rightarrow \cD$. 
Given the domains in \cD, we have a set of associated operators \OP which can be applied 
to the values (\ie, constants) in the domain. 
For each operator $\op \in \OP$, $\arity(\op)$ is its arity, 
and $\domop(\op, i)$ is the domain of the operator for operand $i$, where $1\leq i\leq \arity(\op)$, 
and $\domop(\op)$ is the domain of the result of $\op$.

Our model has an \emph{abstract authorization state}, which we denote by \state, that contains an unbounded number of entities. 
An \emph{entity} is a total map, denoted by $\entity_j$, that maps attribute names to values from their appropriate domains.
Formally, for any $\entity_j$, the following holds: $\forall x \in \cA.\; \entity_j(x) \in \dom(x) \cup \{\bot\}$. 
An example of an entity could an object, a subject, a user, \etc As we want to facilitate existence of unbounded number 
of users, objects, subjects, \etc, in our system, we allow to have unbounded number of entities. 
We denote the set of all possible authorization states as \stateset. 

A \emph{request}, $r$, is a finite sequence of attribute name-value pairs of form $\langle an, av\rangle$ 
where $an\in\cA$ and $av\in\dom(an)$. We represent the set of all requests with \cR. 
Our model also have an additional component, which we call the policy function. A \emph{policy function}, \policyfunc takes 
as argument a request $r$ and an authorization state \state, and returns a pair $\langle d, \state'\rangle$, where $d$ represents the \emph{decision}, 
where $d \in \{\Allow, \Deny\}$, and $\state'$ represents a new authorization state. 
Formally, $\policyfunc : \cR \times \stateset \rightarrow \{\Allow, \Deny\} \times \stateset$. 
The new state $\state'$ reflects any side effects that might have been the result of carrying out the request. For instance, 
when a user requests to create an object and after consulting the previous authorization state \state and the authorization requirement, 
the policy function \policyfunc 
decides to allow the action, it might also create the object, relevant information of which (\eg, attributes, identifier, \etc) 
are stored in the new state $\state'$. 

\section{Specification Language for \policyfunc in \abac} 



\end{document}
